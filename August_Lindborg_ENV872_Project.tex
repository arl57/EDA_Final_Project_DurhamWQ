% Options for packages loaded elsewhere
\PassOptionsToPackage{unicode}{hyperref}
\PassOptionsToPackage{hyphens}{url}
%
\documentclass[
  12pt,
]{article}
\usepackage{lmodern}
\usepackage{amsmath}
\usepackage{ifxetex,ifluatex}
\ifnum 0\ifxetex 1\fi\ifluatex 1\fi=0 % if pdftex
  \usepackage[T1]{fontenc}
  \usepackage[utf8]{inputenc}
  \usepackage{textcomp} % provide euro and other symbols
  \usepackage{amssymb}
\else % if luatex or xetex
  \usepackage{unicode-math}
  \defaultfontfeatures{Scale=MatchLowercase}
  \defaultfontfeatures[\rmfamily]{Ligatures=TeX,Scale=1}
  \setmainfont[]{Times New Roman}
\fi
% Use upquote if available, for straight quotes in verbatim environments
\IfFileExists{upquote.sty}{\usepackage{upquote}}{}
\IfFileExists{microtype.sty}{% use microtype if available
  \usepackage[]{microtype}
  \UseMicrotypeSet[protrusion]{basicmath} % disable protrusion for tt fonts
}{}
\makeatletter
\@ifundefined{KOMAClassName}{% if non-KOMA class
  \IfFileExists{parskip.sty}{%
    \usepackage{parskip}
  }{% else
    \setlength{\parindent}{0pt}
    \setlength{\parskip}{6pt plus 2pt minus 1pt}}
}{% if KOMA class
  \KOMAoptions{parskip=half}}
\makeatother
\usepackage{xcolor}
\IfFileExists{xurl.sty}{\usepackage{xurl}}{} % add URL line breaks if available
\IfFileExists{bookmark.sty}{\usepackage{bookmark}}{\usepackage{hyperref}}
\hypersetup{
  pdftitle={Evaluation of Water Quality in Eno River and Ellerbe Creek Between 2019 and 2020},
  pdfauthor={Olivia August and Analise Lindborg},
  hidelinks,
  pdfcreator={LaTeX via pandoc}}
\urlstyle{same} % disable monospaced font for URLs
\usepackage[margin=2.54cm]{geometry}
\usepackage{longtable,booktabs}
\usepackage{calc} % for calculating minipage widths
% Correct order of tables after \paragraph or \subparagraph
\usepackage{etoolbox}
\makeatletter
\patchcmd\longtable{\par}{\if@noskipsec\mbox{}\fi\par}{}{}
\makeatother
% Allow footnotes in longtable head/foot
\IfFileExists{footnotehyper.sty}{\usepackage{footnotehyper}}{\usepackage{footnote}}
\makesavenoteenv{longtable}
\usepackage{graphicx}
\makeatletter
\def\maxwidth{\ifdim\Gin@nat@width>\linewidth\linewidth\else\Gin@nat@width\fi}
\def\maxheight{\ifdim\Gin@nat@height>\textheight\textheight\else\Gin@nat@height\fi}
\makeatother
% Scale images if necessary, so that they will not overflow the page
% margins by default, and it is still possible to overwrite the defaults
% using explicit options in \includegraphics[width, height, ...]{}
\setkeys{Gin}{width=\maxwidth,height=\maxheight,keepaspectratio}
% Set default figure placement to htbp
\makeatletter
\def\fps@figure{htbp}
\makeatother
\setlength{\emergencystretch}{3em} % prevent overfull lines
\providecommand{\tightlist}{%
  \setlength{\itemsep}{0pt}\setlength{\parskip}{0pt}}
\setcounter{secnumdepth}{5}
\ifluatex
  \usepackage{selnolig}  % disable illegal ligatures
\fi

\title{Evaluation of Water Quality in Eno River and Ellerbe Creek
Between 2019 and 2020}
\usepackage{etoolbox}
\makeatletter
\providecommand{\subtitle}[1]{% add subtitle to \maketitle
  \apptocmd{\@title}{\par {\large #1 \par}}{}{}
}
\makeatother
\subtitle{\url{https://github.com/arl57/EDA_Final_Project_DurhamWQ}}
\author{Olivia August and Analise Lindborg}
\date{}

\begin{document}
\maketitle

\newpage
\tableofcontents 
\newpage
\listoftables 
\newpage
\listoffigures 
\newpage

\hypertarget{rationale-and-research-questions}{%
\section{Rationale and Research
Questions}\label{rationale-and-research-questions}}

Water quality of urban streams and rivers has been studied across the
United States for the past several decades. These evaluations typically
provide important insights about stressors on aquatic systems in urban
environments, particularly natural pollutants such as nutrients,
sediment, and heavy metals, as well as anthropogenic contaminants such
as pesticides and other man-made chemicals. Many cities have programs
dedicated to monitoring the health of local rivers and streams for the
protection of humans and wildlife.

Certain water quality parameters are commonly collected and used to
assess health of urban streams. These include measurements of chemical,
physical, and biological parameters that can be used independently or
together to determine stream health. These parameters are often
evaluated over time to highlight general trends in water quality in
urban environments. In general, health of urban streams in the United
States has been improving since the Clean Water Act was introduced in
the 1970s.

The main objective of our project was to understand current water
quality trends in local urban streams between 2019 and 2020. Eno River
and Ellerbe Creek in Durham, North Carolina were selected for evaluation
in our study. These streams were chosen to provide a local context for
evaluating recent changes in stream health. Additionally, the City of
Durham collects monthly monitoring data for both Eno River and Ellerbe
Creek for several metrics that they provide publicly, making this easily
accessible data. This project includes an evaluation of the most common
surface water quality parameters for which the City of Durham had data,
including temperature, pH, dissolved oxygen, metals (zinc and copper),
total phosphorus, fecal coliform, turbidity, and total suspended solids.
All sites for each water body evaluated by the City of Durham were
included.

\textbf{Research Question:}

\begin{enumerate}
\def\labelenumi{\arabic{enumi}.}
\tightlist
\item
  What are the water quality trends between 2019 and 2020 for Ellerbe
  Creek and Eno River?
\end{enumerate}

\begin{enumerate}
\def\labelenumi{\alph{enumi}.}
\tightlist
\item
  Has water quality changed between 2019 and 2020 for Ellerbe Creek
  and/or Eno River based on the various water quality parameters?
\item
  Are there differences between sites for each stream?
\item
  Have certain water quality parameters changed while others have not?
\item
  Discussion point: Certain studies have shown that surface water
  quality improved in 2020 due to the pandemic. Do we observe the same
  trend?
\end{enumerate}

\newpage

\hypertarget{dataset-information}{%
\section{Dataset Information}\label{dataset-information}}

The data was collected from the City of Durham's water quality data web
portal. The portal includes data collected by the City's Stormwater
Services as part of the Water Quality Monitoring and Assessment Program.
The program includes ambient stream monitoring to assess compliance with
regulatory benchmarks, assess surface water impairment, identify sources
for illicit discharge, and support watershed planning. The monitoring
data includes information regarding the monitoring location, conditions,
weather, and measurements. The nine parameters that were chosen for
analysis had monthly measurements for 2019 and 2020 at the two streams
of interest. The parameters of interest include Copper, Dissolved
Oxygen, Fecal Coliform, pH, Total Phosphorus, Total Suspended Solids,
Temperature, Turbidity, and Zinc. To extract relevant data from the
portal the stream, water quality parameters, and dates of interest were
selected for the user interface and downloaded as a CSV file.

Once datasets for each of the parameters were downloaded, they were read
into R and compiled into a single dataframe. First, a subset of the
dataframe was created to keep only relevant columns for analysis. Next,
the date column to read as a date to enable plotting and time series
analysis. To address duplicate measurements, measurements for the same
parameter, with the same date, monitoring location, and weather were
averaged. Then, the parameters measurements were pivoted to include a
column for each of the nine parameters. To map the locations of each of
the monitoring stations, the water quality data was joined with station
coordinates included in a separate station dataset.

\begin{longtable}[]{@{}llll@{}}
\toprule
Water Quality Parameters & Unit & Range & Data Source\tabularnewline
\midrule
\endhead
Copper & ug/L & 1.1-4.135 & Durham Water Quality Web
Portal\tabularnewline
Dissolved Oxygen & mg/L & 4.7-12.1 & Durham Water Quality Web
Portal\tabularnewline
Fecal Coliform & ug/L & 17.5-36000 & Durham Water Quality Web
Portal\tabularnewline
pH & Standard Units & 6.1-7.5 & Durham Water Quality Web
Portal\tabularnewline
Total Phosphorus & mg/L & 0.003 - 0.38 & Durham Water Quality Web
Portal\tabularnewline
Total Suspended Solids & mg/L & 2.5-134 & Durham Water Quality Web
Portal\tabularnewline
Temperature & C & 5.7-29.6 & Durham Water Quality Web
Portal\tabularnewline
Turbidity & NTU & 2.3-150 & Durham Water Quality Web
Portal\tabularnewline
Zinc & ug/L & 0.975-19.2 & Durham Water Quality Web
Portal\tabularnewline
\bottomrule
\end{longtable}

\newpage

\hypertarget{exploratory-analysis}{%
\section{Exploratory Analysis}\label{exploratory-analysis}}

\newpage

\hypertarget{analysis}{%
\section{Analysis}\label{analysis}}

\hypertarget{question-1-insert-specific-question-here-and-add-additional-subsections-for-additional-questions-below-if-needed}{%
\subsection{Question 1: \textless insert specific question here and add
additional subsections for additional questions below, if
needed\textgreater{}}\label{question-1-insert-specific-question-here-and-add-additional-subsections-for-additional-questions-below-if-needed}}

\hypertarget{question-2}{%
\subsection{Question 2:}\label{question-2}}

\newpage

\hypertarget{summary-and-conclusions}{%
\section{Summary and Conclusions}\label{summary-and-conclusions}}

\newpage

\hypertarget{references}{%
\section{References}\label{references}}

\textless add references here if relevant, otherwise delete this
section\textgreater{}

\end{document}
