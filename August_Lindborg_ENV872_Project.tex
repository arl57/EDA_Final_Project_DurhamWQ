% Options for packages loaded elsewhere
\PassOptionsToPackage{unicode}{hyperref}
\PassOptionsToPackage{hyphens}{url}
%
\documentclass[
  12pt,
]{article}
\usepackage{lmodern}
\usepackage{amsmath}
\usepackage{ifxetex,ifluatex}
\ifnum 0\ifxetex 1\fi\ifluatex 1\fi=0 % if pdftex
  \usepackage[T1]{fontenc}
  \usepackage[utf8]{inputenc}
  \usepackage{textcomp} % provide euro and other symbols
  \usepackage{amssymb}
\else % if luatex or xetex
  \usepackage{unicode-math}
  \defaultfontfeatures{Scale=MatchLowercase}
  \defaultfontfeatures[\rmfamily]{Ligatures=TeX,Scale=1}
  \setmainfont[]{Times New Roman}
\fi
% Use upquote if available, for straight quotes in verbatim environments
\IfFileExists{upquote.sty}{\usepackage{upquote}}{}
\IfFileExists{microtype.sty}{% use microtype if available
  \usepackage[]{microtype}
  \UseMicrotypeSet[protrusion]{basicmath} % disable protrusion for tt fonts
}{}
\makeatletter
\@ifundefined{KOMAClassName}{% if non-KOMA class
  \IfFileExists{parskip.sty}{%
    \usepackage{parskip}
  }{% else
    \setlength{\parindent}{0pt}
    \setlength{\parskip}{6pt plus 2pt minus 1pt}}
}{% if KOMA class
  \KOMAoptions{parskip=half}}
\makeatother
\usepackage{xcolor}
\IfFileExists{xurl.sty}{\usepackage{xurl}}{} % add URL line breaks if available
\IfFileExists{bookmark.sty}{\usepackage{bookmark}}{\usepackage{hyperref}}
\hypersetup{
  pdftitle={Evaluation of Water Quality in Eno River and Ellerbe Creek Between 2019 and 2020},
  pdfauthor={Olivia August and Analise Lindborg},
  hidelinks,
  pdfcreator={LaTeX via pandoc}}
\urlstyle{same} % disable monospaced font for URLs
\usepackage[margin=2.54cm]{geometry}
\usepackage{longtable,booktabs}
\usepackage{calc} % for calculating minipage widths
% Correct order of tables after \paragraph or \subparagraph
\usepackage{etoolbox}
\makeatletter
\patchcmd\longtable{\par}{\if@noskipsec\mbox{}\fi\par}{}{}
\makeatother
% Allow footnotes in longtable head/foot
\IfFileExists{footnotehyper.sty}{\usepackage{footnotehyper}}{\usepackage{footnote}}
\makesavenoteenv{longtable}
\usepackage{graphicx}
\makeatletter
\def\maxwidth{\ifdim\Gin@nat@width>\linewidth\linewidth\else\Gin@nat@width\fi}
\def\maxheight{\ifdim\Gin@nat@height>\textheight\textheight\else\Gin@nat@height\fi}
\makeatother
% Scale images if necessary, so that they will not overflow the page
% margins by default, and it is still possible to overwrite the defaults
% using explicit options in \includegraphics[width, height, ...]{}
\setkeys{Gin}{width=\maxwidth,height=\maxheight,keepaspectratio}
% Set default figure placement to htbp
\makeatletter
\def\fps@figure{htbp}
\makeatother
\setlength{\emergencystretch}{3em} % prevent overfull lines
\providecommand{\tightlist}{%
  \setlength{\itemsep}{0pt}\setlength{\parskip}{0pt}}
\setcounter{secnumdepth}{5}
\ifluatex
  \usepackage{selnolig}  % disable illegal ligatures
\fi

\title{Evaluation of Water Quality in Eno River and Ellerbe Creek
Between 2019 and 2020}
\usepackage{etoolbox}
\makeatletter
\providecommand{\subtitle}[1]{% add subtitle to \maketitle
  \apptocmd{\@title}{\par {\large #1 \par}}{}{}
}
\makeatother
\subtitle{\url{https://github.com/arl57/EDA_Final_Project_DurhamWQ}}
\author{Olivia August and Analise Lindborg}
\date{}

\begin{document}
\maketitle

\newpage
\tableofcontents 
\newpage
\listoftables 
\newpage
\listoffigures 
\newpage

\hypertarget{rationale-and-research-questions}{%
\section{Rationale and Research
Questions}\label{rationale-and-research-questions}}

Water quality of urban streams and rivers has been studied across the
United States for the past several decades. These evaluations typically
provide important insights about stressors on aquatic systems in urban
environments, particularly natural pollutants such as nutrients,
sediment, and heavy metals, as well as anthropogenic contaminants such
as pesticides and other man-made chemicals. Many cities have programs
dedicated to monitoring the health of local rivers and streams for the
protection of humans and wildlife.

Certain water quality parameters are commonly collected and used to
assess health of urban streams. These include measurements of chemical,
physical, and biological parameters that can be used independently or
together to determine stream health. These parameters are often
evaluated over time to highlight general trends in water quality in
urban environments. In general, health of urban streams in the United
States has been improving since the Clean Water Act was introduced in
the 1970s.

The main objective of our project was to understand current water
quality trends in local urban streams between 2019 and 2020. Eno River
and Ellerbe Creek in Durham, North Carolina were selected for evaluation
in our study. These streams were chosen to provide a local context for
evaluating recent changes in stream health. Additionally, the City of
Durham collects monthly monitoring data for both Eno River and Ellerbe
Creek for several metrics that they provide publicly, making this easily
accessible data. This project includes an evaluation of the most common
surface water quality parameters for which the City of Durham had data,
including temperature, pH, dissolved oxygen, metals (zinc and copper),
total phosphorus, fecal coliform, turbidity, and total suspended solids.
All sites for each water body evaluated by the City of Durham were
included.

\textbf{Research Question:}

\begin{enumerate}
\def\labelenumi{\arabic{enumi}.}
\tightlist
\item
  What are the water quality trends between 2019 and 2020 for Ellerbe
  Creek and Eno River?

  \begin{enumerate}
  \def\labelenumii{\alph{enumii}.}
  \tightlist
  \item
    Has water quality changed between 2019 and 2020 for Ellerbe Creek
    and/or Eno River based on the various water quality parameters?
  \item
    Are there differences between sites for each stream?
  \item
    Have certain water quality parameters changed while others have not?
  \item
    Discussion point: Certain studies have shown that surface water
    quality improved in 2020 due to the pandemic. Do we observe the same
    trend?
  \end{enumerate}
\end{enumerate}

\newpage

\hypertarget{dataset-information}{%
\section{Dataset Information}\label{dataset-information}}

The data was collected from the City of Durham's water quality data web
portal. The portal includes data collected by the City's Stormwater
Services as part of the Water Quality Monitoring and Assessment Program.
The program performs ambient stream monitoring to assess compliance with
regulatory benchmarks, assess surface water impairment, identify sources
for illicit discharge, and support watershed planning. The monitoring
data includes information regarding the monitoring location, conditions,
weather, and measurements. Nine parameters were chosen for analysis
because of their monthly measurement frequency for 2019 and 2020 at the
two streams of interest. The parameters of interest include Copper,
Dissolved Oxygen, Fecal Coliform, pH, Total Phosphorus, Total Suspended
Solids, Temperature, Turbidity, and Zinc. To extract relevant data from
the portal the stream, water quality parameters, and dates of interest
were selected for the user interface and downloaded to CSV files.

Once datasets for each of the parameters were downloaded, they were read
into R and compiled into a single dataframe. First, a subset of the
dataframe was created to keep only relevant columns for analysis. Next,
the date column to read as a date to enable plotting and time series
analysis. To address duplicate measurements, measurements for the same
parameter, with the same date and monitoring location were averaged.
Then, the parameters measurements were pivoted to include a column for
each of the nine parameters. To map the locations of each of the
monitoring stations, the water quality dataframe was joined with station
coordinates.

\begin{longtable}[]{@{}llll@{}}
\toprule
\begin{minipage}[b]{(\columnwidth - 3\tabcolsep) * \real{0.49}}\raggedright
Water Quality Parameters\strut
\end{minipage} &
\begin{minipage}[b]{(\columnwidth - 3\tabcolsep) * \real{0.12}}\raggedright
Unit\strut
\end{minipage} &
\begin{minipage}[b]{(\columnwidth - 3\tabcolsep) * \real{0.14}}\raggedright
Range\strut
\end{minipage} &
\begin{minipage}[b]{(\columnwidth - 3\tabcolsep) * \real{0.25}}\raggedright
Data Source\strut
\end{minipage}\tabularnewline
\midrule
\endhead
\begin{minipage}[t]{(\columnwidth - 3\tabcolsep) * \real{0.49}}\raggedright
Copper\strut
\end{minipage} &
\begin{minipage}[t]{(\columnwidth - 3\tabcolsep) * \real{0.12}}\raggedright
ug/L\strut
\end{minipage} &
\begin{minipage}[t]{(\columnwidth - 3\tabcolsep) * \real{0.14}}\raggedright
1.1-4.135\strut
\end{minipage} &
\begin{minipage}[t]{(\columnwidth - 3\tabcolsep) * \real{0.25}}\raggedright
Durham Water Quality Web Portal\strut
\end{minipage}\tabularnewline
\begin{minipage}[t]{(\columnwidth - 3\tabcolsep) * \real{0.49}}\raggedright
Dissolved Oxygen\strut
\end{minipage} &
\begin{minipage}[t]{(\columnwidth - 3\tabcolsep) * \real{0.12}}\raggedright
mg/L\strut
\end{minipage} &
\begin{minipage}[t]{(\columnwidth - 3\tabcolsep) * \real{0.14}}\raggedright
4.7-12.1\strut
\end{minipage} &
\begin{minipage}[t]{(\columnwidth - 3\tabcolsep) * \real{0.25}}\raggedright
Durham Water Quality Web Portal\strut
\end{minipage}\tabularnewline
\begin{minipage}[t]{(\columnwidth - 3\tabcolsep) * \real{0.49}}\raggedright
Fecal Coliform\strut
\end{minipage} &
\begin{minipage}[t]{(\columnwidth - 3\tabcolsep) * \real{0.12}}\raggedright
cfu/100mL\strut
\end{minipage} &
\begin{minipage}[t]{(\columnwidth - 3\tabcolsep) * \real{0.14}}\raggedright
17.5-36000\strut
\end{minipage} &
\begin{minipage}[t]{(\columnwidth - 3\tabcolsep) * \real{0.25}}\raggedright
Durham Water Quality Web Portal\strut
\end{minipage}\tabularnewline
\begin{minipage}[t]{(\columnwidth - 3\tabcolsep) * \real{0.49}}\raggedright
pH\strut
\end{minipage} &
\begin{minipage}[t]{(\columnwidth - 3\tabcolsep) * \real{0.12}}\raggedright
Standard Units\strut
\end{minipage} &
\begin{minipage}[t]{(\columnwidth - 3\tabcolsep) * \real{0.14}}\raggedright
6.1-7.5\strut
\end{minipage} &
\begin{minipage}[t]{(\columnwidth - 3\tabcolsep) * \real{0.25}}\raggedright
Durham Water Quality Web Portal\strut
\end{minipage}\tabularnewline
\begin{minipage}[t]{(\columnwidth - 3\tabcolsep) * \real{0.49}}\raggedright
Total Phosphorus\strut
\end{minipage} &
\begin{minipage}[t]{(\columnwidth - 3\tabcolsep) * \real{0.12}}\raggedright
mg/L\strut
\end{minipage} &
\begin{minipage}[t]{(\columnwidth - 3\tabcolsep) * \real{0.14}}\raggedright
0.003 - 0.38\strut
\end{minipage} &
\begin{minipage}[t]{(\columnwidth - 3\tabcolsep) * \real{0.25}}\raggedright
Durham Water Quality Web Portal\strut
\end{minipage}\tabularnewline
\begin{minipage}[t]{(\columnwidth - 3\tabcolsep) * \real{0.49}}\raggedright
Total Suspended Solids\strut
\end{minipage} &
\begin{minipage}[t]{(\columnwidth - 3\tabcolsep) * \real{0.12}}\raggedright
mg/L\strut
\end{minipage} &
\begin{minipage}[t]{(\columnwidth - 3\tabcolsep) * \real{0.14}}\raggedright
2.5-134\strut
\end{minipage} &
\begin{minipage}[t]{(\columnwidth - 3\tabcolsep) * \real{0.25}}\raggedright
Durham Water Quality Web Portal\strut
\end{minipage}\tabularnewline
\begin{minipage}[t]{(\columnwidth - 3\tabcolsep) * \real{0.49}}\raggedright
Temperature\strut
\end{minipage} &
\begin{minipage}[t]{(\columnwidth - 3\tabcolsep) * \real{0.12}}\raggedright
C\strut
\end{minipage} &
\begin{minipage}[t]{(\columnwidth - 3\tabcolsep) * \real{0.14}}\raggedright
5.7-29.6\strut
\end{minipage} &
\begin{minipage}[t]{(\columnwidth - 3\tabcolsep) * \real{0.25}}\raggedright
Durham Water Quality Web Portal\strut
\end{minipage}\tabularnewline
\begin{minipage}[t]{(\columnwidth - 3\tabcolsep) * \real{0.49}}\raggedright
Turbidity\strut
\end{minipage} &
\begin{minipage}[t]{(\columnwidth - 3\tabcolsep) * \real{0.12}}\raggedright
NTU\strut
\end{minipage} &
\begin{minipage}[t]{(\columnwidth - 3\tabcolsep) * \real{0.14}}\raggedright
2.3-150\strut
\end{minipage} &
\begin{minipage}[t]{(\columnwidth - 3\tabcolsep) * \real{0.25}}\raggedright
Durham Water Quality Web Portal\strut
\end{minipage}\tabularnewline
\begin{minipage}[t]{(\columnwidth - 3\tabcolsep) * \real{0.49}}\raggedright
Zinc\strut
\end{minipage} &
\begin{minipage}[t]{(\columnwidth - 3\tabcolsep) * \real{0.12}}\raggedright
ug/L\strut
\end{minipage} &
\begin{minipage}[t]{(\columnwidth - 3\tabcolsep) * \real{0.14}}\raggedright
0.975-19.2\strut
\end{minipage} &
\begin{minipage}[t]{(\columnwidth - 3\tabcolsep) * \real{0.25}}\raggedright
Durham Water Quality Web Portal\strut
\end{minipage}\tabularnewline
\begin{minipage}[t]{(\columnwidth - 3\tabcolsep) * \real{0.49}}\raggedright
Rain in the last 24 Hours\strut
\end{minipage} &
\begin{minipage}[t]{(\columnwidth - 3\tabcolsep) * \real{0.12}}\raggedright
NA\strut
\end{minipage} &
\begin{minipage}[t]{(\columnwidth - 3\tabcolsep) * \real{0.14}}\raggedright
Yes/No\strut
\end{minipage} &
\begin{minipage}[t]{(\columnwidth - 3\tabcolsep) * \real{0.25}}\raggedright
Durham Water Quality Web Portal\strut
\end{minipage}\tabularnewline
\begin{minipage}[t]{(\columnwidth - 3\tabcolsep) * \real{0.49}}\raggedright
Sky Condition\strut
\end{minipage} &
\begin{minipage}[t]{(\columnwidth - 3\tabcolsep) * \real{0.12}}\raggedright
NA\strut
\end{minipage} &
\begin{minipage}[t]{(\columnwidth - 3\tabcolsep) * \real{0.14}}\raggedright
Sunny, Partly Cloudy, Overcast\strut
\end{minipage} &
\begin{minipage}[t]{(\columnwidth - 3\tabcolsep) * \real{0.25}}\raggedright
Durham Water Quality Web Portal\strut
\end{minipage}\tabularnewline
\bottomrule
\end{longtable}

\newpage

\hypertarget{exploratory-analysis}{%
\section{Exploratory Analysis}\label{exploratory-analysis}}

As part of the exploratory analysis, water quality data was compiled for
the nine parameters for both the Eno River and Ellerbe Creek. The
following sections describe the exploratory analysis completed for each
waterway.

\hypertarget{eno-river}{%
\section{\#\# Eno River}\label{eno-river}}

Both Ellerbe Creek and Eno River have three monitoring stations in
Durham. Sites EN13.3ER, EN8.9ER, and EN4.9ER for Eno River all had data
for 2019, but only \textbf{\emph{(insert sites)}} had data for 2020. For
Ellerbe Creek, sites EL1.9EC, EL5.6EC, and EL7.1EC had data for 2019,
but only EL1.9EC and EL7.1EC had data for 2020.

\begin{figure}

{\centering \includegraphics[width=1\linewidth]{SiteMap} 

}

\caption{Site Map of Ellerbe Creek and Eno River Monitoring Stations}\label{fig:unnamed-chunk-1}
\end{figure}

\hypertarget{ellerbe-creek}{%
\subsection{Ellerbe Creek}\label{ellerbe-creek}}

Visual explorations of the nine water quality parameters for Ellerbe
Creek were conducted to determine how these parameters changed between
2019 and 2020. First, because we are not able to visit the sites, we
wanted to determine what the main site characteristics may be. One main
exploration was the relationship between temperature and cloud cover,
hypothesizing that if temperature is fluctuating with changes in cloud
cover (e.g., temperatures increase on sunny days), this could indicated
minimal riparian vegetation. Minimal riparian vegetation has impacts on
water quality such as increased sedimentation, temperature, and runoff
contamination. At Ellerbe Creek, for all stations, it was observed that
temperature does not appear to change directly in response to cloud
cover (Figure 1). Do make any definitive determinations about this
relationship, we would need more finite data than just one sample day
per month.

\includegraphics{August_Lindborg_ENV872_Project_files/figure-latex/unnamed-chunk-4-1.pdf}

We were also interested in the relationship between turbidity and rain,
as an indicator of erosion and to what degree the Ellerbe Creek sites
may be influenced by storm events. Visual comparisons show that there
appears to be an increase in turbidity after rain events, especially in
2020 (Figure 2). Figure 2 also reveals that there may be an increase in
turbidity between 2019 and 2020.
\includegraphics{August_Lindborg_ENV872_Project_files/figure-latex/unnamed-chunk-5-1.pdf}

After completing site explorations, we wanted to look specifically at
water quality parameters between 2019 and 2020.

\textbf{\emph{ADD individual figures}}

\hypertarget{eno-river-1}{%
\subsection{Eno River}\label{eno-river-1}}

\hypertarget{ellerbe-creek-1}{%
\subsection{Ellerbe Creek}\label{ellerbe-creek-1}}

\newpage

\hypertarget{analysis}{%
\section{Analysis}\label{analysis}}

\hypertarget{zinc-copper}{%
\subsection{Zinc \& Copper}\label{zinc-copper}}

\hypertarget{ph}{%
\subsection{pH}\label{ph}}

\hypertarget{turbidity-and-recent-rainfall}{%
\subsection{Turbidity and Recent
Rainfall}\label{turbidity-and-recent-rainfall}}

\hypertarget{tss-and-turbidity}{%
\subsection{TSS and Turbidity}\label{tss-and-turbidity}}

Correlated parameters.

\hypertarget{temperature-and-dissolved-oxygen}{%
\subsection{Temperature and Dissolved
Oxygen}\label{temperature-and-dissolved-oxygen}}

Discuss impact of Temperature on DO. Similar Temp from 2019-2020.

\hypertarget{temperature-and-fecal-coliform}{%
\subsection{Temperature and Fecal
Coliform}\label{temperature-and-fecal-coliform}}

\hypertarget{question-1-insert-specific-question-here-and-add-additional-subsections-for-additional-questions-below-if-needed}{%
\subsection{Question 1: \textless insert specific question here and add
additional subsections for additional questions below, if
needed\textgreater{}}\label{question-1-insert-specific-question-here-and-add-additional-subsections-for-additional-questions-below-if-needed}}

\hypertarget{question-2}{%
\subsection{Question 2:}\label{question-2}}

\newpage

\hypertarget{summary-and-conclusions}{%
\section{Summary and Conclusions}\label{summary-and-conclusions}}

\newpage

\hypertarget{references}{%
\section{References}\label{references}}

\textless add references here if relevant, otherwise delete this
section\textgreater{}

\end{document}
